\newpage
\section{Use Cases}
  in diesem Kapitel werden die Akteure beschrieben (Kapitel 2.1), das Use-Case-Diagramm wird gezeigt (Kapitel 2.2) und die einzelnen Use Cases werden detailliert beschrieben (Kapitel 2.3 ff.)
  
\subsection{Akteure}
  \subsubsection{Mitarbeiter}
    Unter Mitarbeiter fallen P"adagogen und Praktikanten, die in der Kinderkrippe arbeiten.
    
  \subsubsection{aktiver Elternteil}
    Unter aktiven Elternteilen versteht man Personen, die mindestens ein Kind in der Kinderkrippe angemeldet haben.
    
  \subsubsection{inaktiver Elternteil}
    Unter Inaktive Elternteilen versteht man Personen, deren Kind aus der Kinderkrippe abgemeldet wurde.
    
  \subsubsection{System}  
    
 \newpage
 \subsection{Use-Case-Diagramm}
 \begin{figure}[ht!]
  \includegraphics[width = 150mm]{pictures/usecasediagram.jpg}
 \end{figure}

 
 \newpage
 \subsection{Use-Case  Anmelden}
  \paragraph{Identifier}
    UC 2.01
  \paragraph{Beschreibung}
    Um diese Applikation nutzen zu k"onnen, muss sich der Nutzer mit Username und Passwort anmelden.
  \paragraph{Ausl"oser}
    Der Nutzer gibt in den daf"ur vorgesehenen Schaltfl"achen den Benutzernamen und das zugeh"orige Passwort ein und best"atigt seine Eingaben mit dem Button \dq Login \dq.
  \paragraph{Beteiligte Akteure}   \leavevmode \newline
    Mitarbeiter, inaktiver Elternteil, aktiver Elternteil
  \paragraph{Vorbedingungen}
  \begin{itemize}
   \item Es existiert ein g"ultiger Account
   \item Der Nutzer befindet sich in Maske M1-Login
  \end{itemize}

  \paragraph{Schritte}
  \begin{enumerate}
   \item User gibt seinen Username und sein Passwort ein.
   \item Der Benutzer best"atigt seine Eingaben mit dem Button \dq Login\dq
   \item Das System "uberpr"uft die Daten
   \item Der User wird auf die Maske M2-Grundstellung weitergeleitet
  \end{enumerate}

  \paragraph{Alternative Schritte}
  \begin{enumerate}
   \setcounter{enumi}{3}
   \item Die eingegebenen Daten sind nicht korrekt. Der User bekommt eine entsprechende Fehlermeldung
  \end{enumerate}

  \paragraph{Nachbedingungen}
    Der Benutzer befindet sich im angemeldeten Zustand und befindet sich in Maske M2-Grundstellung.

  
 \newpage
 \subsection{Use-Case Abmelden}
  \paragraph{Identifier}
  UC 2.02
  \paragraph{Beschreibung}
  Die aktuelle Sitzung zwischen Benutzer und System wird beendet.
  \paragraph{Ausl"oser}
  \begin{itemize}
   \item Der Benutzer klickt auf den Button \dq Logout\dq
   \item Es hat f"ur einen gewissen Zeitraum keine Eingabe gegeben
   \item Die Applikation wird beendet
  \end{itemize}

  \paragraph{Beteiligte Akteure}   \leavevmode \newline
    Mitarbeiter, inaktiver Elternteil, aktiver Elternteil
  \paragraph{Vorbedingungen}
  \begin{itemize}
   \item Der Benutzer ist angemeldet
  \end{itemize}

  \paragraph{Schritte}
  \begin{enumerate}
   \item Der Benutzer klickt auf den Button \dq Logout\dq
   \item Das System "andert den Status des Benutzers auf \dq abgemeldet\dq
   \item Der Benutzer wird auf die Login Seite M1-Login weitergeleitet
  \end{enumerate}

  \paragraph{Alternative Schritte}
  \begin{enumerate}
  \setcounter{enumi}{0}
   \item Die Applikation wird geschlossen
   \setcounter{enumi}{0}
   \item Der Benutzer hat f"ur eine gewisse Zeit keine Eingabe get"atigt
  \end{enumerate}

  \paragraph{Nachbedingungen}
  Alle aktiven Sitzungen wurden beendet und der Benutzer befindet sich in Maske M1-Login

 \newpage
 \subsection{Use-Case Passwort vergessen}
  \paragraph{Identifier}
  UC 2.03
  \paragraph{Beschreibung}
  Falls der Benutzer das Passwort vergessen hat, kann er sich ein zufällig generiertes Passwort an seine Emailadresse senden lassen.
  \paragraph{Ausl"oser}
  \begin{itemize}
   \item Der Benutzer klickt auf den Button \dq Passwort vergessen\dq
  \end{itemize}

  \paragraph{Beteiligte Akteure}   \leavevmode \newline
    Mitarbeiter, inaktiver Elternteil, aktiver Elternteil
  \paragraph{Vorbedingungen}
  \begin{itemize}
   \item Der Benutzer ist angemeldet
  \end{itemize}

  \paragraph{Schritte}
  \begin{enumerate}
   \item Der Benutzer klickt auf den Button \dq Passwort vergessen\dq
   \item Der Benutzer gibt in das dafür vorgesehene Feld die Emailadresse ein und bestätigt diese
   \item Dem Benutzer wird eine Email mit neuem Passwort zugesendet.
  \end{enumerate}

  \paragraph{Alternative Schritte}
  \paragraph{Nachbedingungen}
  Alle aktiven Sitzungen wurden beendet und der Benutzer befindet sich in Maske M1-Login
  
 \newpage
 \subsection{Use-Case Tagesbelegung eintragen}
  \paragraph{Identifier}
  UC 2.04
  \paragraph{Beschreibung}
  Die Mitarbeiter der Kinderkrippe k"onnen nur eine gewisse Anzahl Kinder in Abh"angigkeit der anwesenden P"adagogen und Praktikanten aufnehmen. Diese Anzahl wird gespeichert mit den Bring und Abholzeiten.
  \paragraph{Ausl"oser}
  Der Mitarbeiter klickt auf den Button \dq Krippeninfo\dq
  \paragraph{Beteiligte Akteure}   \leavevmode \newline
    Mitarbeiter
  \paragraph{Vorbedingungen}
  \begin{itemize}
   \item Der Mitarbeiter ist angemeldet
  \end{itemize}

  \paragraph{Schritte}
  \begin{enumerate}
   \item Der Mitarbeiter klickt auf den Button \dq Information hinzufügen\dq
   \item Es "offnet sich ein Pop-up Fenster
   \item Der Benutzer gibt die gew"unschten Eingaben ein und best"atigt mit \dq OK\dq
   \item Die Daten für den gewählten Tag wurden gespeichert
  \end{enumerate}

  \paragraph{Alternative Schritte}
  \begin{enumerate}
  \setcounter{enumi}{3}
   \item Die Eingabe ist ung"ultig und der Benutzer erh"alt eine entsprechende Fehlermeldung
  \end{enumerate}

  \paragraph{Nachbedingungen}

  
  \newpage
 \subsection{Use-Case Feiertage und Ferienzeiten eintragen}
  \paragraph{Identifier}
  UC 2.05
  \paragraph{Beschreibung}
  Der Mitarbeiter kann Feiertage und kinderkrippenspezifische Ferientage in den Kalender eintragen.
  \paragraph{Ausl"oser}
  Der Mitarbeiter klickt auf den Button \dq Kalender\dq
  \paragraph{Beteiligte Akteure}   \leavevmode \newline
    Mitarbeiter
  \paragraph{Vorbedingungen}
  \begin{itemize}
   \item der Mitarbeiter ist angemeldet
  \end{itemize}

  \paragraph{Schritte}
  \begin{enumerate}
   \item Der Benutzer klickt auf den Button \dq Kalender\dq
   \item Der Benutzer wird auf Kalender weitergeleitet
   \item Der Benutzer kann "uber das Kalendermen"u den gew"unschten Tag ausw"ahlen
   \item Es "offnet sich ein Pop-up Fenster
   \item Der Mitarbeiter gibt den gew"unschten Grund und optional eine Anmerkung hinzu
   \item Das System speichert die Daten und der Benutzer wird auf die Hauptseite weitergeleitet
  \end{enumerate}

  \paragraph{Alternative Schritte}
  \paragraph{Nachbedingungen}
  Der Feiertag/Ferientag ist f"ur alle auf Makse M3-Kalender sichtbar

  
 \newpage
 \subsection{Use-Case Kind anmelden}
  \paragraph{Identifier}
  UC 2.06
  \paragraph{Beschreibung}
    Die Mitarbeiter k"onnen "uber eine Schaltfl"ache neue Kinder hinzuf"ugen.
  \paragraph{Ausl"oser}
    Der Benutzer klickt auf den Button \dq Kinder\dq
  \paragraph{Beteiligte Akteure}   \leavevmode \newline
    Mitarbeiter
  \paragraph{Vorbedingungen}
  \begin{itemize}
   \item Der Mitarbeiter ist angemeldet
   \item Ein Elternteil des Kindes ist registriert
  \end{itemize}

  \paragraph{Schritte}
  \begin{enumerate}
   \item Der Mitarbeiter sucht "uber die Suchfunktion nach dem Elternteil
   \item Der Mitarbeiter w"ahlt das Elternteil aus und akzeptiert die Eingabe.
   \item Der Benutzer klickt auf den Button \dq Kind hinzuf"ugen\dq
   \item Es "offnet sich ein Pop-up Fenster
   \item Der Mitarbeiter gibt alle verpflichtenden Felder ein
   \item Der Mitarbeiter gibt eventuell optionale Felder ein
   \item Der Mitarbeiter best"atigt seine Eingaben
   \item Das System legt ein neues Kind in der Datenbank an, falls das Elternteil inaktiv war, wird es auf aktiv gesetzt
  \end{enumerate}

  \paragraph{Alternative Schritte}
  \begin{enumerate}
  \setcounter{enumi}{5}
   \item Die eingegeben Daten waren nicht korrekt, dem Benutzer bekommt eine Fehlermeldung
  \end{enumerate}

  \paragraph{Nachbedingungen}
  Der Benutzer befindet sich auf Maske M2-Grundstellung und das Kind befindet sich in der Datenbank

  \newpage
 \subsection{Use-Case Kind abmelden}
  \paragraph{Identifier}
  UC 2.07
  \paragraph{Beschreibung}
  Das Kind kann von der Kinderkrippe abgemeldet werden.
  \paragraph{Ausl"oser}
  \begin{itemize}
   \item Das Person meldet das Kind manuell ab
  \end{itemize}

  \paragraph{Beteiligte Akteure}   \leavevmode \newline
    Mitarbeiter
  \paragraph{Vorbedingungen}
  \begin{itemize}
   \item Der Mitarbeiter ist angemeldet
   \item Das Kind ist angemeldet
  \end{itemize}

  \paragraph{Schritte}
  \begin{enumerate}
   \item Der Mitarbeiter sucht "uber die Suchfunktion das Kind
   \item Der Benutzer best"atigt seine Auswahl
   \item Der Benutzer klickt auf den Button \dq Kind abmleden\dq
   \item Es "offnet sich ein Pop-up Fenster mit einer Abmeldebest"atigung
   \item Der Mitarbeiter best"atigt mit dem Button \dq OK\dq
   \item Das System l"oscht das Kind und alle korrelierten Kontakte, ist kein kind mehr des Elternteils angemeldet, so wird das Elternteil auf inaktiv gesetzt
  \end{enumerate}

  \paragraph{Alternative Schritte}
  \paragraph{Nachbedingungen}
  Der Benutzer befindet sich auf Maske M2-Grundstellung und das Kind wurde abgemeldet

  
  \newpage
 \subsection{Use-Case Benutzerdaten "andern}
  \paragraph{Identifier}
  UC 2.08
  \paragraph{Beschreibung}
  Der Mitarbeiter kann die Daten von Eltern und Kinder "andern oder aktualisieren.
  \paragraph{Ausl"oser}
  Der Mitarbeiter klickt auf den Button \dq Kinder \dq oder \dq Elternteil \dq
  \paragraph{Beteiligte Akteure}   \leavevmode \newline
    Mitarbeiter
  \paragraph{Vorbedingungen}
  \begin{itemize}
   \item Der Mitarbeiter ist angemeldet
   \item Der zu "andernde User ist registriert
  \end{itemize}

  \paragraph{Schritte}
  \begin{enumerate}
   \item Der Mitarbeiter sucht "uber die Suchfunktion die zu "andernde Person
   \item Der Benutzer best"atigt sein Auswahl
   \item Der Benutzer klickt auf den Button \dq Nutzerdaten "andern\dq
   \item Es "offnet sich ein Pop-up Fenster
   \item Der Mitarbeiter "andert die gew"unschten Felder
   \item Der Mitarbeiter best"atigt sine Eingabe
   \item Das System speichert die Daten und wechselt zu Maske M2-Grundstellung
  \end{enumerate}

  \paragraph{Alternative Schritte}
  \begin{enumerate}
  \setcounter{enumi}{4}
   \item Die eingegebenen Daten sind ung"ultig und der Mitarbeiter bekommt eine entsprechende Fehlermeldung
  \end{enumerate}

  \paragraph{Nachbedingungen}
  Die Nutzerdaten wurden ge"andert und der Benutzer befindet sich auf Maske M2-Grundstellung

  
  \newpage
 \subsection{Use-Case Daten drucken}
  \paragraph{Identifier}
  UC 2.09
  \paragraph{Beschreibung}
  Es k"onnen Tagesplaner, Stammbl"atter, Kontaktlisten, Auditlog, Essenstabellen und Aufgabenlisten ausgedruckt werden
  \paragraph{Ausl"oser}
    Der Benutzer klickt aud den Button \dq Drucken\dq
  \paragraph{Beteiligte Akteure}   \leavevmode \newline
    Mitarbeiter, aktiver Elternteil
  \paragraph{Vorbedingungen}
  \begin{itemize}
   \item Der Benutzer ist angemeldet
   \item Der Benutzer befindet sich auf einer der folgenden Seiten: Kinder, Eltern, Tages/Wochen/Monats/Jahresplaner, Mittagessen, Auditlog, Bezugspersonen
  \end{itemize}

  \paragraph{Schritte}
  \begin{enumerate}
   \item Der Benutzer klickt auf den Button \dq Drucken \dq
   \item Es "offnet sich ein Pop-up Fenster
   \item Der Benutzer w"ahlt den Drucker aus und best"atigt seine Auswahl
   \item Das System sendet einen Druckerauftrag und wechselt zu M2-Grundstellung
  \end{enumerate}
  \paragraph{Alternative Schritte}
  \paragraph{Nachbedingungen}
  Der Nutzer befindet sich in Maske M2-Grundstellung

  
  \newpage
 \subsection{Use-Case Daten exportieren}
  \paragraph{Identifier}
  UC 2.10
  \paragraph{Beschreibung}
    Es k"onnen Tagesplaner, Stammbl"atter, Kontaktlisten, Auditlog, Kalender und Essensbesllung in CSV/PDF/EXCEL/JSON exportiert werden
  \paragraph{Ausl"oser}
    Der Benutzer klickt auf den Button \dq Daten exportieren\dq

  \paragraph{Beteiligte Akteure}   \leavevmode \newline
    Mitarbeiter, aktiver Elternteil
  \paragraph{Vorbedingungen}
    \begin{itemize}
   \item Der Benutzer ist angemeldet
   \item Der Benutzer befindet sich auf  einer der folgenden Seiten: Kinder, Eltern, Tages/Wochen/Monats/Jahresplaner, Mittagessen, Auditlog, Bezugspersonen
  \end{itemize}
  \paragraph{Schritte}
  \begin{enumerate}
   \item Der Benutzer klickt auf den Button \dq Daten exportieren \dq
   \item Es "offnet sich ein Pop-up Fenster
   \item Das System konvertiert und speichert die Daten in einem dafür entsprechenden Ordner
  \end{enumerate}
  \paragraph{Alternative Schritte}
  \paragraph{Nachbedingungen}
  Der Nutzer befindet sich in Maske M2-Grundstellung

  
  \newpage
 \subsection{Use-Case Elternteil registrieren}
  \paragraph{Identifier}
  UC 2.11
  \paragraph{Beschreibung}
  Wenn ein neues Elternteil dem Verein beitritt, kann es von einem Mitarbeiter registriert werden
  \paragraph{Ausl"oser}
  Der Mitarbeiter klickt auf den Button \dq Elternteil\dq
  \paragraph{Beteiligte Akteure}   \leavevmode \newline
    Mitarbeiter
  \paragraph{Vorbedingungen}
  \begin{itemize}
   \item Der Mitarbeiter ist angemeldet
  \end{itemize}
  \paragraph{Schritte}
  \begin{enumerate}
   \item Der Benutzer klickt auf den Button \dq Elternteil registrieren\dq
   \item Der Mitarbeiter gibt alle verpflichtenden Felder ein
   \item Der Mitarbeiter best"atigt seine Eingaben
   \item Das System legt einen neuen inaktiven Nutzer an
  \end{enumerate}

  \paragraph{Alternative Schritte}
  \begin{enumerate}
  \setcounter{enumi}{1}
   \item Die eingegebenen Daten sind ung"ultig und der Nutzer erh"alt eine Fehlermeldung
  \end{enumerate}

  \paragraph{Nachbedingungen}
  Der neue Nutzer befindet sich in der Datenbank
    
	
	\newpage
	\subsection{Use-Case Anlegen von Bezugspersonen}
	  \paragraph{Identifier}
  UC 2.12
		\paragraph{Beschreibung}
		Ein Elternteil kann eine Bezugsperson für sein Kind anlegen
		\paragraph{Ausl"oser}
		Der Elternteil klickt auf den Button \dq Bezugsperson hinzuf"ugen\dq
		\paragraph{Beteiligte Akteure}   \leavevmode \newline
		Elternteil
		\paragraph{Vorbedingungen}
			\begin{itemize}
			 	\item Der Elternteil ist angemeldet
			\end{itemize}
		
		\paragraph{Schritte}
			\begin{enumerate}
			 	\item Der Elternteil klickt auf den Button \dq Bezugsperson hinzuf"ugen\dq
			 		\item Es "offnet sich ein Pop-up Fenster in welchem die Daten der Bezugsperson eingetragen werden können
			 	\item Der Elternteil tragt die Daten der Bezugsperson ein
			 	\item Der Elternteil best"atigt die Anlegung mit einem Klick auf den Button \dq Best"atigen\dq 
			\end{enumerate}
		
		\paragraph{Alternative Schritte}
		Der Elternteil bricht die Anlegung einer Bezugsperson ab. Das Pop-up Fenster wird wieder geschlossen und es werden keine Daten in der Datenbank gespeichert. 	
		\paragraph{Nachbedingungen}
		Die Bezugsperson ist in der Datenbank gespeichert, aber noch nicht im System freigegeben. Die Freischaltung einer Bezugsperson kann daraufhin über einen Mitarbeiter erfolgen. 
  
	\newpage
	\subsection{Use-Case Freischaltung der Bezugspersonen}
	  \paragraph{Identifier}
  UC 2.13
		\paragraph{Beschreibung}
		Mitarbeitern ist es möglich die angelegten Bezusgpersonen der Kinder der Elternteile freizuschalten
		\paragraph{Ausl"oser}
		Der Elternteil klickt auf den Button \dq Bezuspgerson ansehen\dq
		\paragraph{Beteiligte Akteure}   \leavevmode \newline
		Mitarbeiter
		\paragraph{Vorbedingungen}
		\begin{itemize}
		 	\item Der Mitarbeiter ist angemeldet
		\end{itemize}
		
		\paragraph{Schritte}
		\begin{enumerate}
		 	\item Der Mitarbeiter klickt auf den Button \dq Bezugsperson ansehen\dq
		 	\item Es "offnet sich ein Pop-up Fenster mit den Daten der Person
		 	\item der Mitarbeiter best"atigt seine Auswahl mit einem Klick auf den Button \dq best"atigen\dq
		\end{enumerate}
		
		\paragraph{Alternative Schritte}
		\begin{enumerate}
			\item Der Mitarbeiter klickt auf den Button \dq Bezugspersonen freischalten\dq
			\item Es "offnet sich ein Pop-up Fenster mit allen freizuschaltenden Bezugspersonen
			\item Der Mitarbeiter bricht die Freischaltung der Bezugspersonen mit einem Klick auf den Butt \dq Abbrechen \dq ab
		\end{enumerate}
			
		\paragraph{Nachbedingungen}
		Die Bezugspersonen sind im System freigeschaltet. Diese Änderung ist auch in der Datenbank geändert worden. 
  
  
  \newpage
 \subsection{Use-Case Kinderstammblatt einsehen (Mitarbeiter)}
  \paragraph{Identifier}
  UC 2.14
  \paragraph{Beschreibung}
  In dieser Ansicht erh"alt der Mitarbeiter die vollst"andigen Information "uber das Kind
  \paragraph{Ausl"oser}
  Der Mitarbeiter klickt auf den Button \dq Kinderdaten ansehenn\dq
  \paragraph{Beteiligte Akteure}   \leavevmode \newline
    Mitarbeiter
  \paragraph{Vorbedingungen}
  \begin{itemize}
   \item Der Mitarbeiter ist angemeldet
   \item Mindestens ein Kind ist in der Kinderkrippe registriert
  \end{itemize}

  \paragraph{Schritte}
  \begin{enumerate}
   \item Der Mitarbeiter sucht "uber die Suchfunktion das Kind
   \item Der Mitarbeiter best"atigt seine Auswahl
   \item Der Mitarbeiter klickt auf die entsprechende Tabellenzeile.
   \item Es "offnet sich ein Pop-up Fenster mit allen Informationen und einem Button \dq Stammblatt PDF \dq
   \item Der Benutzer klickt den Button \dq Stammblatt PDF \dq
   \item Das System erzeugt eine PDF mit allen Informationen über das Kind
  \end{enumerate}
  
  \paragraph{Alternative Schritte}
  \paragraph{Nachbedingungen}

  
  \newpage
 \subsection{Use-Case Planer einsehen}
  \paragraph{Identifier}
  UC 2.15
  \paragraph{Beschreibung}
  In dieser Ansicht erh"alt der Mitarbeiter eine kompakte "Ubersicht "uber die geplanten T"atigkeiten des gew"ahlten Zeitraumes
  \paragraph{Ausl"oser}
  Der Mitarbeiter klickt auf einen der Button: "Tagesplaner einsehen", "Wochenplaner einsehen", "Monatsplaner einsehen", "Jahresplaner einsehen"
  \paragraph{Beteiligte Akteure}   \leavevmode \newline
    Mitarbeiter
  \paragraph{Vorbedingungen}
  \begin{itemize}
   \item Der Benutzer ist angemeldet
   \item Der Benuzer befindet sich auf der Kalenderseite
  \end{itemize}

  \paragraph{Schritte}
  \begin{enumerate}
   \item Der Benutzer klickt in Maske M3-Kalender auf den gew"unschten Tag, um den Tagesplaner zu "offnen
   \item Der Benutzer klickt auf den Button \dq Wochenplaner einsehen\dq , um den Wochenplaner zu "offnen
   \item Der Benutzer klickt auf den Button \dq Monatsplaner einsehen\dq , um den Monatsplaner zu "offnen
   \item Der Benutzer klickt auf den Button \dq Jahresplaner einsehen\dq , um den Jahresplaner zu "offnen
   \item Der Benutzer wird auf die gewählte Seite weitergeleitet
  \end{enumerate}

  \paragraph{Alternative Schritte}
  \paragraph{Nachbedingungen}
  Der Benutzer befindet sich auf der gewählten Seite

 
 \newpage
 \subsection{Use-Case Aufgaben der Eltern eintragen}
  \paragraph{Identifier}
  UC 2.16
  \paragraph{Beschreibung}
  Die Mitarbeiter der Kinderkrippe k"onnen anfallende Arbeiten Elternteile zuweisen
  \paragraph{Ausl"oser}
  Der Benutzer klickt auf den Kalender den jeweiligen Tag an
  \paragraph{Beteiligte Akteure}   \leavevmode \newline
    Mitarbeiter
  \paragraph{Vorbedingungen}
  \begin{itemize}
   \item Der Benutzer ist angemeldet
   \item Der Benutzer befindet sich auf der Seite Kalender
  \end{itemize}

  \paragraph{Schritte}
  \begin{enumerate}
  \item Der Benutzer klickt in Maske M3-Kalender auf den gew"unschten Tag
  \item Es "offnet sich ein Pop-up Fenster
  \item der Benutzer gibt alle notwendigen Daten ein (Name, Aufgabe, Zeipunkt, Zeitraum)
  \item Der Benutzer best"atigt seine Eingabe
  \item Das System sendet eine Benachrichtigung an das Elternteil und wechselt zu M2-Grundstellung
  \end{enumerate}
  
	\newpage
	\subsection{Use-Case Einsehen der eigenen Aufgaben}
	\paragraph{Beschreibung}
	Elternteile können die ihnen zugeteilten Aufgaben einsehen 
	\paragraph{Ausl"oser}
	Der Elternteil klickt auf den Button \dq Aufgaben ansehen\dq
	\paragraph{Beteiligte Akteure}   \leavevmode \newline
	Elternteil
	\paragraph{Vorbedingungen}
	\begin{itemize}
		\item Der Elternteil ist angemeldet
	\end{itemize}
	
	\paragraph{Schritte}
	\begin{enumerate}
		\item Der Elternteil klickt auf den Button \dq Eigene Aufgaben ansehen \dq
		\item Es "offnet sich ein Pop-up Fenster in dem alle dem Elternteil zugeteilten Aufgaben vermerkt sind
	\end{enumerate}
  
	\paragraph{Alternative Schritte}
	\paragraph{Nachbedingungen}

  
  \newpage
 \subsection{Use-Case Auditlog einsehen}
  \paragraph{Identifier}
  UC 2.17
  \paragraph{Beschreibung}
  Der Auditlog ist eine unver"anderliche Logdatei, die alle Aktionen chronologisch mit Zeitstempel speichert. Auf diese kann "uber unsere Applikation lesend zugegriffen werden 
  \paragraph{Ausl"oser}
  Der Benutzer klickt auf den Button \dq Auditlog lesen\dq
  \paragraph{Beteiligte Akteure}   \leavevmode \newline
    Mitarbeiter
  \paragraph{Vorbedingungen}
  \begin{itemize}
   \item Der Nutzer ist angemeldet
  \end{itemize}

  \paragraph{Schritte}
  \begin{enumerate}
   \item Der Mitarbeiter klickt auf den Button \dq Auditlog lesen\dq
   \item Das System wechselt zur Maske M4-Auditlog
  \end{enumerate}
  \paragraph{Alternative Schritte}
  \paragraph{Nachbedingungen}
  Der Nutzer befindet sich in Maske M4-Auditlog

  
  \newpage
 \subsection{Use-Case Bring- und Abholzeiten eintragen (Eltern)}
  \paragraph{Identifier}
  UC 2.18
  \paragraph{Beschreibung}
  Um den Mitarbeitern die Zeiteinteilung zu erleichtern, k"onnen Eltern die Bring- und Abholzeiten eintragen.
  \paragraph{Ausl"oser}
  Der Benutzer klickt auf den Button \dq Bringen/Abholen\dq
  \paragraph{Beteiligte Akteure}   \leavevmode \newline
    aktiver Elternteil
  \paragraph{Vorbedingungen}
  \begin{itemize}
   \item Der Benutzer ist angemeldet
   \item Der Benutzer befindet sich in Maske M3-Kalender
  \end{itemize}

  \paragraph{Schritte}
  \begin{enumerate}
   \item Der Benutzer klickt in M3-Kalender den gew"unschten Tag an
   \item Der Benutzer klickt auf den Button \dq Bringen/Abholen\dq
   \item Es "offnet sich ein Pop-up Fenster
   \item Der Benutzer gibt die Daten (Bringzeit, Abholzeit, Kind, Selbstabholung oder Fremdabholung) ein und best"atigt seine Eingaben
   \item Das System speichert die Daten und wechselt zur Maske M2-Grundstellung
  \end{enumerate}

  \paragraph{Alternative Schritte}
  \begin{enumerate}
  \setcounter{enumi}{4}
   \item  Die Eingaben sind ung"ultig und der Benutzer erh"alt eine Fehlermeldung
  \end{enumerate}

  \paragraph{Nachbedingungen}
  Die Bring und Abholzeiten werden in der Datenbank gespeichert und der Nutzer befindet sich in M2-Grundstellung
 
  \newpage
 \subsection{Use-Case Mittagessen anmelden/abmelden}
  \paragraph{Identifier}
  UC 2.19
  \paragraph{Beschreibung}
  Das Kind kann w"ochentlich f"ur das Mittagessen angemeldet werden. Eine Abmeldung ist bis zum Stichtag m"oglich.
  \paragraph{Ausl"oser}
  Der Benutzer klickt auf den Button Mitagessen Anmelden/Abmelden
  \paragraph{Beteiligte Akteure}   \leavevmode \newline
    aktiver Elternteil
  \paragraph{Vorbedingungen}
  \begin{itemize}
   \item Der Benutzer ist angemeldet
   \item Der Benutzer befindet sich in M3-Kalender
  \end{itemize}

  \paragraph{Schritte}
  \begin{enumerate}
   \item Der Benutzer klickt in M3-Kalender auf den gew"unschten Tag
   \item Der Benutzer klickt auf den Button \dq Mittagessen Anmelden/Abmelden\dq
   \item Es "offnet sich ein Pop-up Fenster
   \item Der Benutzer gibt die Wochentage f"ur die folgende Wochen ein, in der das Kind Mittagessen konsumiert
   \item Der Benutzer best"atigt die Eingabe
   \item Das System speichert die Daten und der User wird auf M2-Grundstellung weitergeleitet
  \end{enumerate}

  \paragraph{Alternative Schritte}
  \begin{enumerate}
  \setcounter{enumi}{2}
   \item  Wenn der Benutzer an diesem Tag f"ur ein Mittagessen angemeldet ist, wird die ANmeldung an diesem Tag storniert.
   \item Das System speichert die "Anderung
  \end{enumerate}

  \paragraph{Nachbedingungen}
  Die Anmeldung befindet sich in der Datenbank

  
    \newpage
 \subsection{Use-Case Eigene Daten bearbeiten}
  \paragraph{Identifier}
  UC 2.20
  \paragraph{Beschreibung}
  Die Anwender der Applikation k"onnen jederezeit ihre personenbezogenen Daten "andern
  \paragraph{Ausl"oser}
  Der Benutzer klickt auf den Button \dq Eigene Daten bearbeiten\dq
  \paragraph{Beteiligte Akteure}   \leavevmode \newline
    Mitarbeiter, aktiver Elternteil, inaktiver Benutzer
  \paragraph{Vorbedingungen}
  \begin{itemize}
   \item Der Benutzer ist angemeldet
  \end{itemize}

  \paragraph{Schritte}
  \begin{enumerate}
   \item Der Benutzer klickt auf den Button \dq Eingen Daten bearbeiten\dq
   \item Es "offnet sich ein Pop-up Fenster
   \item Der Benutzer kann seine Daten "andern
   \item Der Benutzer klickt auf den Button \dq annehmen\dq
   \item Das System speichert die "Anderung
  \end{enumerate}

  \paragraph{Alternative Schritte}
  \begin{enumerate}
   \item Die eingegebenen Daten sind ung"ultig und der User erh"alt eine Fehlermeldung
  \end{enumerate}

  \paragraph{Nachbedingungen}
 
	\newpage
	\subsection{Use-Case Anmeldung für E-Mail-Benachrichtigungen}
	  \paragraph{Identifier}
  UC 2.21
	\paragraph{Beschreibung}
	Aktive Benutzer des Systems können sich für E-Mail-Benachrichtigungen anmelden. 
	\paragraph{Ausl"oser}
	Der Benutzer klickt auf den Button \dq E-Mail-Benachrichtigungen\dq
	\paragraph{Beteiligte Akteure}   \leavevmode \newline
	Mitarbeiter, aktiver Elternteil
	\paragraph{Vorbedingungen}
	\begin{itemize}
	 	\item Der Benutzer ist angemeldet
	\end{itemize}
	
	\paragraph{Schritte}
	\begin{enumerate}
	 	\item Der Benutzer klickt auf den Button \dq E-Mail-Benachrichtigungen\dq
		\item Es "offnet sich ein Pop-up mit einem Best"atigungs- und einem Abbruch-Button
	 	\item Der Benutzer best"atigt das Abonnement mit einem Klick auf den Button \dq Best"atigen\dq
	\end{enumerate}
	
	\paragraph{Alternative Schritte}
	\begin{enumerate}
		\item Der Benutzer klickt auf den Button \dq E-Mail-Benachrichtigungen\dq
		\item Es "offnet sich ein Pop-up mit einem Best"atigungs- und einem Abbruch-Button
		\item Der Benutzer bricht den Vorgang mit einem Klick auf den Button \dq Abbrechen\dq ab
	\end{enumerate}
	\paragraph{Nachbedingungen}
	Die "Anderungen sind in der Datenbank gespeichert und der Benutzer erh"alt von nun an alle E-Mail-Benachrichtigungen. 
	
	
	
	
	\newpage
	\subsection{Use-Case Abmeldung von E-Mail-Benachrichtigungen}
	  \paragraph{Identifier}
  UC 2.22
	\paragraph{Beschreibung}
	Aktive Benutzer des Systems können sich von den E-Mail-Benachrichtigungen, für welche sie sich angemolden haben, wieder abmelden 
	\paragraph{Ausl"oser}
	Der Benutzer klickt auf den Button \dq E-Mail-Benachrichtigungen\dq
	\paragraph{Beteiligte Akteure}   \leavevmode \newline
	Mitarbeiter, aktiver Elternteil
	\paragraph{Vorbedingungen}
	\begin{itemize}
		\item Der Benutzer ist angemeldet
		\item Der Benutzer hat sich für E-Mail-Benachrichtigungen angemolden
	\end{itemize}
	
	\paragraph{Schritte}
	\begin{enumerate}
		\item Der Benutzer klickt auf den Button \dq E-Mail-Benachrichtigungen\dq
		\item Es "offnet sich ein Pop-up mit einem Abmelde- und einem Abbruch-Button
		\item Der Benutzer best"atigt die Abmeldung mit einem Klick auf den Button \dq Best"atigen\dq
	\end{enumerate}
	
	\paragraph{Alternative Schritte}
	\begin{enumerate}
		\item Der Benutzer klickt auf den Button \dq E-Mail-Benachrichtigungen\dq
		\item Es "offnet sich ein Pop-up mit einem Abmelde- und einem Abbruch-Button
		\item Der Benutzer bricht die Abmeldung mit einem Klick auf den Button \dq Abbrechen\dq ab
	\end{enumerate}
	\paragraph{Nachbedingungen}
	Die "Anderungen sind in der Datenbank gespeichert und der Benutzer erh"alt von nun an keine E-Mail-Benachrichtigungen mehr. 
  
  
  \newpage
 \subsection{Use-Case Messageboard "offnen}
  \paragraph{Identifier}
  UC 2.23
  \paragraph{Beschreibung}
  Der Benutzer "offnet das Messageboard mit Klick auf einen Button. Es öffnet sich eine Seite mit chronologisch angeordneten Textfeld-Komponenten (Neueste zuerst). 
  \paragraph{Ausl"oser}
  Der Benutzer klickt auf den Button \dq Pinnwand \dq
  \paragraph{Beteiligte Akteure}   \leavevmode \newline
    Mitarbeiter, aktiver Elternteil
  \paragraph{Vorbedingungen}
	\begin{itemize}
		\item Der Benutzer ist angemeldet
	\end{itemize}
  \paragraph{Schritte}
  	\begin{enumerate}
		\item Der Benutzer klickt auf den Button \dq Pinnwand\dq
		\item Der Benutzer wird auf die Seite Pinnwand weitergeleitet
	\end{enumerate}
  \paragraph{Alternative Schritte}
  \paragraph{Nachbedingungen}

  
  \newpage
 \subsection{Use-Case Nachricht auf Messageboard posten}
  \paragraph{Identifier}
  UC 2.24
  \paragraph{Beschreibung}
  Über das Eingabefeld können neue Nachrichten (Textfeld-Komponenten) erstellt werden. Diese Nachricht wird für jeden aktiven Elternteil sichtbar.
  \paragraph{Ausl"oser}
  \paragraph{Beteiligte Akteure}   \leavevmode \newline
    Mitarbeiter, aktiver Elternteil
  \paragraph{Vorbedingungen}
  	\begin{itemize}
		\item Der Benutzer ist angemeldet
	\end{itemize}
  \paragraph{Schritte}
  	\begin{enumerate}
		\item Der Benutzer klickt auf den Button \dq Pinnwand\dq
		\item Der Benutzer wird auf die Seite Pinnwand weitergeleitet
		\item Der Benutzer klickt auf den Button \dq Nachricht senden \dq
		\item Es "offnet sich ein Fenster, in das die Nachricht eingegeben werden kann
		\item Der Benutzer best"atigt seine Eingabe
	\end{enumerate}
  \paragraph{Alternative Schritte}
  \paragraph{Nachbedingungen}
  
  \newpage
 \subsection{Use-Case Privatnachricht senden (Sonderfunktion)}
  \paragraph{Identifier}
  UC 2.25
  \paragraph{Beschreibung}
  Aktive Mitglieder k"onnen sich untereinander Privatnachrichten senden
  \paragraph{Ausl"oser}
  Der Benutzer klickt auf den Button \dq Privatnachricht \dq
  \paragraph{Beteiligte Akteure}   \leavevmode \newline
    Mitarbeiter, aktiver Elternteil
  \paragraph{Vorbedingungen}
    	\begin{itemize}
		\item Der Benutzer ist angemeldet
	\end{itemize}
  \paragraph{Schritte}
    	\begin{enumerate}
		\item Der Benutzer klickt auf den Button \dq Privatnachricht\dq
		\item Es "offnet sich ein Fenster, in dem der Benutzer ausgew"ahlt werden kann
		\item Der Benutzer best"atigt seine Eingabe
		\item Es "offnet sich ein Fenster, in dem die Privatnachricht eingegeben werden kann
	\end{enumerate}
  \paragraph{Alternative Schritte}
  \paragraph{Nachbedingungen}